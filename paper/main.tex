\documentclass[twocolumn]{aastex62}
\usepackage{natbib}
\bibliographystyle{apj}
\begin{document}



%=========================================================================
%		FRONT MATTER
%=========================================================================
\title{Quantifying and controlling biases in dark matter halo concentration estimates}
\author{
  ASd, Asd, ASd
}

\affil{
$^1$Somewhere.\\
$^2$Elsewhere
}



\begin{abstract}
We assign fibers to targets.
\end{abstract}

\keywords{
  Galaxies: halos --- Dark matter --- Methods: numerical 
}


%*************************************************************************
\section{Introduction}
\label{sec:introduction}

Massive Spectroscopic surveys aiming at constructing three-dimensional
maps of the large-scale structure of the Universe are key experiments
in the prograss of cosmology and fundamental physics.

One of the key tasks is to match the position of fibers to the images
of galaxies in the focal plane. This assignment process is constrained
by the hardware's geometry, the location of the galaxies on the focal
plane and the scientific goals in the survey. 
These three aspects guide the fiber assignment algorithmic dessign.
In practice this design have grown after an iterative process since
the earliest prototipes in 2013
\url{https://github.com/forero/FiberAllocation}.   

\cite{2014SPIE.9150E..23S}
\cite{2016A&C....15....1N}

\bibliography{references}
\end{document}
